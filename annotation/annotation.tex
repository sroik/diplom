% !TEX root = ./annotation.tex

\documentclass [a4paper, 12pt]{report}
\usepackage [utf8]{inputenc}
\usepackage[T2A]{fontenc}
\usepackage [russianb] {babel}
\usepackage {amsfonts,amssymb,eucal,amsmath,latexsym}
\usepackage{graphics}
\usepackage{array}
\usepackage{multirow}
\usepackage[unicode]{hyperref}

\voffset=-14mm
\textheight=23cm
\linespread{1.3}
\oddsidemargin=5mm
\textwidth=16cm

\usepackage{listings}
\lstloadlanguages{Python}

\begin {document}

\thispagestyle{empty}
\begin{normalsize}
	\begin{center}
		{\bf МИНИСТЕРСТВО ОБРАЗОВАНИЯ РЕСПУБЛИКИ БЕЛАРУСЬ}
	\end{center}

	\begin{center}
		{\bf БЕЛОРУCСКИЙ ГОСУДАРСТВЕННЫЙ УНИВЕРСИТЕТ}
	\end{center}

	\begin{center}
		{\bf Факультет прикладной математики и информатики}
	\end{center}

	\begin{center}
		Кафедра математического моделирования и анализа данных
	\end{center}
\end{normalsize}
\bigskip
\bigskip
\bigskip
\bigskip
\bigskip
\bigskip

\begin{center}
	Аннотация к дипломной работе \\
	{\bf Приближенное вычисление математического ожидания функционалов от гауссовского процесса, основанное на интерполяции корреляционной функции}
\end{center}
\bigskip
\bigskip
\bigskip
\bigskip

\begin{center}
	Кожановский Василий Николаевич
\end{center}

\bigskip
\bigskip
\bigskip
\bigskip

\begin{center}
	Научный руководитель – доктор физ.-мат. наук, профессор А.Д. Егоров
\end{center}

\vspace{\stretch{1.5}}
\bigskip
\bigskip
\bigskip
\bigskip

\begin{center}
	\bf{Минск 2017}
\end{center}

\newpage

\begin{center}
	\textbf{АННОТАЦИЯ}
\end{center}

\emph{\textbf{Димпломная работа}}, 30 стр., 2 рис., 3 табл., 1 приложение, 5 источников.

\emph{\textbf{Ключевые слова:}} ГАУССОВСКИЕ ПРОЦЕССЫ, ФОРМУЛА ИНТЕРПОЛЯЦИИ, АПРОКСИМАЦИЯ МАТЕМАТИЧЕСКИХ ОЖИДАНИЙ, МАТЕМАТИЧЕСКОЕ ОЖИДАНИЕ, ФУНКЦИОНАЛЫ ОТ ПРОЦЕССА.

\emph{\textbf{Объект исследования}} -- математическое ожидание функционалов от гауссовского процесса.

\emph{\textbf{Цель работы}} -- пременение формулы интерполяции корреляционной функции гауссовского процесса к приближенному вычислению функционалов от процессов.

\emph{\textbf{Методы исследования}} -- методы вычислительной математики, теория случайных процессов.

\emph{\textbf{Результатом}} являются полученные оценки точности метода вычисления мат. ожиданий функционалов.

\emph{\textbf{Областью применения}} является аппроксимация математических ожиданий функционалов от гауссовских процессов.

\bigskip
\bigskip

\begin{center}
	\textbf{ABSTRACT}
\end{center}

\emph{\textbf{Graduation assignment}}, 30 p., 2 pic., 3 tables, 1 app, 5 sources.

\emph{\textbf{Keywords:}} GAUSSIAN PROCESSES, INTERPOLATION FORMULA,\\ MATHEMATICAL EXPECTATION, FUNCTIONALS.

\emph{\textbf{Research object}} -- mathematical expectation from gaussian processes.

\emph{\textbf{Purpose of the work}} -- explore the approximation of the mathematical expectations of functionals of solutions of stochastic differential equations.

\emph{\textbf{Research methods}} -- methods of computational mathematics, chaotic decomposition.

\emph{\textbf{The result}} is obtained estimates of the accuracy of the method of calculating the math. functional expectations.

\emph{\textbf{Sphere of application }}is approximation of the mathematical expectations of functionals.

\end {document}
